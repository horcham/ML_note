\section{线性回归与梯度下降}
\subsection{符号说明}
\begin{center}
\begin{tabular}{cc}
\toprule[2pt]
变量 & 含义 \\ 
\midrule[1pt]
$x^{(i)}$ & 第$i$个特征输入(features) \\ 
$y^{(i)}$ & 第$i$个目标输出(target) \\ 
$(x^{(i)},y^{(i)})$ & 第$i$个训练样本 \\ 
$\mathbf{X}$ & 输入向量空间,为列向量 \\ 
$\mathbf{y}$ & 输出变量 \\ 
$m$ & 数据样本个数 \\
$n$ & 输入变量的特征个数\\
$\theta$ & 参数向量,为列向量\\
\bottomrule[2pt]
\end{tabular}
\end{center}


\subsection{线性回归}
考虑线性函数
\begin{eqnarray}
h_\theta(x)=\theta_0+\theta_1x_1+\theta_2x_2+\cdots+\theta_nx_n
\end{eqnarray}
为了简化$h_\theta(x)$,将$x_0=1$引入,令$\mathbf{\theta(x)}=(\theta_0,\theta_1,\cdots,\theta_n)^T$,则有
\begin{eqnarray}
h_\theta(x)
&=&\sum_{i=0}^n\theta_ix_i\\
&=&
\begin{pmatrix}
\theta_0&\theta_1&\cdots & \theta_n
\end{pmatrix}
\begin{pmatrix}
1\\
x_1\\
\vdots\\
x_n
\end{pmatrix}\\
&=&\mathbf{\theta}^Tx
\end{eqnarray}
定义损失函数
\begin{eqnarray}
J(\theta)=\frac{1}{2}\sum_{i=1}^m(h_\theta(x^{(i)})-y^{(i)})^2
\end{eqnarray}
该方法被称为普通最小二乘方法(ordinary least squares)。其中,增加系数$\frac{1}{2}$是为了求导方便,而不影响其方向。
\subsection{LMS与梯度下降法}
为了通过最小化误差函数$J(\theta)$来求解最优的$\theta$。其中一个方法是采用梯度下降法。梯度下降法采用迭代$\theta$来使得$J(\theta)$变小,直到达到一定的值之后停止迭代。该算法任意初始化一个$\theta$,通过如下方式更新
\begin{eqnarray}
\theta_j := \theta_j-\alpha\frac{\partial}{\partial \theta_j}J(\theta)
\end{eqnarray}
其中,$\alpha$为学习率(learning rate),决定每次迭代的调整尺度。若对单个的训练样本$(x^{(i)},y^{(i)})$,计算$J(\theta)$的偏导,有
\begin{eqnarray}
\frac{\partial}{\partial \theta_j}
&=&\frac{\partial}{\partial \theta_j}\frac{1}{2}(h_\theta(x^{(i)})-y^{(i)})^2\\
&=&(h_\theta(x^{(i)})-y^{(i)})\frac{\partial}{\partial\theta_j}(h_\theta(x^{(i)})-y^{(i)})\\
&=&(h_\theta(x)-y^{(i)})\frac{\partial}{\partial\theta_j}
\left(
\begin{aligned}
\sum_{i=1}^n\theta_ix^{(i)}_i-y^{(i)}
\end{aligned}
\right)\\
&=&(h_\theta(x^{(i)})-y^{(i)})x_j
\end{eqnarray}
因而,其更新规则为
\begin{eqnarray}
\theta_j:=\theta_j+\alpha(y^{(i)}-h_\theta(x^{(i)}))x^{(i)}_j
\end{eqnarray}
该规则被称为LMS(least mean squares)。在迭代过程中,通常不会选择大的学习率,因其会产生过拟合的情况,或产生在最优解附近震荡。一般情况下,对$J(\theta)$的最优值求解是二次规划问题,是有最优解的。假设$m$是数据样本个数,梯度下降法考虑遍历所有数据样本之后,在对参数$\mathbf{\theta}$进行更新,公式如下
\begin{eqnarray}
\theta_j:=\theta_j+\alpha\sum_{i=1}^m(y^{(i)}-h_\theta(x^{(i)}))x_j^{(i)}
\end{eqnarray}
由于这种更新方式需要遍历所有的数据样本,其迭代速度会比较慢,以下两种方法是对梯度下降法的改进算法。
\subsection{随机梯度下降法(incremental gradient descent)}
在样本中随机抽取一个样本,并采取式(11)调整$\theta_j$,循环直到误差值符合收敛条件。这种算法是对梯度下降法的改进,其对抽取的一个样本便进行调整,而无需式(12)进行遍历,计算速度会显著加快。但因其采用片面的数据作调整,其调整的梯度有可能并非是最优方向,因而其有可能会陷入局部最优解,且需要更多次迭代来达到收敛条件。
\subsection{批量梯度下降法(batch gradient descnet)}
这是一种对于梯度下降法和随机梯度下降法的这种方法。假设采集批量为$p$,则在数据集中做$p$次采样,拓展成一个有$p$个样本的数据集,通过式(12)来更新$\theta$。此方法常会取得好的效果。
\subsection{线性回归的显式求解}
设矩阵$A=
\begin{pmatrix}
a_{11} & \cdots & a_{1n}\\
\vdots & \ddots & \vdots\\
a_{m1} & \cdots & a_{mn}
\end{pmatrix}
$
,
$B=
\begin{pmatrix}
b_{11} & \cdots & b_{1n}\\
\vdots & \ddots & \vdots\\
b_{m1} & \cdots & b_{mn}
\end{pmatrix}
$
,$\textbf{tr}(A)$为矩阵$A$的迹,即对角线上元素之和,$a\in \mathbb{R}$,函数$f$定义为$f(A)=AB$,
则有如下结论
\begin{eqnarray}
\textbf{tr}AB&=&\textbf{tr}BA\\
\textbf{tr}ABC&=&\textbf{tr}CAB = \textbf{tr}BCA\\
\textbf{tr}A&=&\textbf{tr}A^T\\
\textbf{tr}(A+B)&=&\textbf{tr}A+\textbf{tr}B\\
\textbf{tr}aA&=&a\textbf{tr}A\\
\nabla_{A}\textbf{tr}AB &=& B^T\\
\nabla_A \textbf{tr}(ABA^TC) &=& CAB+C^TAB^T\\
\nabla_{A^T}f(A)&=&(\nabla_Af(A))^T\\
\nabla_A|A|&=&|A|(A^{-1})
\end{eqnarray}
我们将第$i$个样本的输入记为$x^{(i)}$,其为列向量:
\begin{eqnarray}
x^{(i)}=(x^{(i)}_1,x^{(i)}_2,\cdots,x^{(i)}_n)^T
\end{eqnarray}
则样本输入可表示为
\begin{eqnarray}
X=
\begin{pmatrix}
(x^{(1)})^T\\
(x^{(2)})^T\\
\vdots\\
(x^{(n)})^T
\end{pmatrix}
\end{eqnarray}
将样本输出记为
\begin{eqnarray}
Y=
\begin{pmatrix}
y^{(1)}\\
y^{(2)}\\
\vdots\\
y^{(n)}
\end{pmatrix}
\end{eqnarray}

由于$\theta$为列向量,因而有
\begin{eqnarray}
\begin{aligned}
X\theta-Y
&=
\begin{pmatrix}
(x^{(1)})^T\theta\\
(x^{(2)})^T\theta\\
\vdots\\
(x^{(m)})^T\theta
\end{pmatrix}
-
\begin{pmatrix}
y^{(1)}\\
y^{(2)}\\
\vdots\\
y^{(m)}
\end{pmatrix}\\
&=
\begin{pmatrix}
(x^{(1)})^T\theta-y^{(1)}\\
(x^{(2)})^T\theta-y^{(2)}\\
\vdots\\
(x^{(m)})^T\theta-y^{(m)}
\end{pmatrix}
\end{aligned}
\end{eqnarray}
由于对于列向量$z$,有$z^Tz=\sum_iz_i^2$,因而有
\begin{eqnarray}
\begin{aligned}
J(\theta)&=\frac{1}{2}\sum_{i=1}^m(h_\theta(x^{(i)})-y^{(i)})^2\\
&= \frac{1}{2}(X\theta-Y)^T(X\theta-Y)\end{aligned}
\end{eqnarray}
因而有
\begin{eqnarray}
\begin{aligned}
\nabla_\theta J(\theta)&=\nabla_\theta\frac{1}{2}\sum_{i=1}^m(h_\theta(x^{(i)})-y^{(i)})^2\\
&=\frac{1}{2}\nabla_\theta((X\theta)^T(X\theta)-(X\theta)^TY-Y^T(X\theta)-Y^TY)\\
&=\frac{1}{2}\nabla_\theta(\theta^TX^TX\theta-\theta^TX^TY-Y^TX\theta-Y^TY)\\\end{aligned}
\end{eqnarray}
由于$\theta^TX^TX\theta-\theta^TX^TY-Y^TX\theta-Y^TY$为数,即$1\times1$的矩阵,因而有
\begin{eqnarray}
\begin{aligned}
\theta^TX^TX\theta-\theta^TX^TY-Y^TX\theta-Y^TY=\textbf{tr}(\theta^TX^TX\theta-\theta^TX^TY-Y^TX\theta-Y^TY)
\end{aligned}
\end{eqnarray}
所以有
\begin{eqnarray}
\begin{aligned}
\nabla_\theta J(\theta)
 &=\frac{1}{2}\nabla_\theta(\theta^TX^TX\theta-\theta^TX^TY-Y^TX\theta-Y^TY)\\
&=\frac{1}{2}\nabla_\theta\textbf{tr}(\theta^TX^TX\theta-\theta^TX^TY-Y^TX\theta-Y^TY)\\
&=\frac{1}{2}\nabla_\theta(\textbf{tr}\theta^TX^TX\theta-\textbf{tr}\theta^TX^TY-\textbf{tr}Y^TX\theta+\textbf{tr}Y^TY)
\end{aligned}
\end{eqnarray}
由于$\textbf{tr}AB=\textbf{tr}BA$,因而有$\textbf{tr}\theta^TX^TY=\textbf{tr}Y^TX\theta$,且$\nabla_\theta\textbf{tr}Y^TY=0$,于是有
\begin{eqnarray}
\begin{aligned}
\nabla_\theta J(\theta)&=\frac{1}{2}\nabla_\theta(\textbf{tr}\theta^TX^TX\theta-2\textbf{tr}Y^TX\theta)\\
&=\frac{1}{2}(X^TX\theta+X^TX\theta-2X^TY)\\
&=X^TX\theta-X^TY
\end{aligned}
\end{eqnarray}
令$\nabla_\theta J(\theta)=X^TX\theta-X^TY=0$,则可得到
$X^TX\theta-X^TY$
一般情况下,$X^TX$为可逆矩阵,因而有
\begin{eqnarray}
\theta=(X^TX)^{-1}X^TY
\end{eqnarray}