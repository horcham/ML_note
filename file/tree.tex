\section{决策树}
决策树学习,假设给定训练数据集
\begin{eqnarray}
D=\{ (\sample{x}{1},\sample{y}{1}),(\sample{x}{2},\sample{y}{2}),\cdots,(\sample{x}{N},\sample{y}{N}) \}
\end{eqnarray}
其中,$\sample{x}{i}=(\samplet{x}{i}{1},\samplet{x}{i}{2},\cdots,\samplet{x}{i}{n})$,$n$为特征个数,$\sample{y}{i}\in\{ 1,2,\cdots,K \}$,$K$为类别数目,$i=1,2,\cdots,N$,$N$为样本容量。
\subsection{特征选择}
\subsubsection{信息增益}
设$X$是一个取有限个值的离散随机变量,其概率分布为
\begin{eqnarray}
P(X=\sample{x}{i})=p_i,i=1,2,\cdots,n
\end{eqnarray}
则随机变量$X$的熵定义为
\begin{eqnarray}
H(X)=-\sum_{i=1}^n p_i\log p_i
\end{eqnarray}
熵越大,随机变量的不确定性就越大。

设有随机变量$(X,Y)$,其联合概率分布为
\begin{eqnarray}
P(X=\sample{x}{i},Y=\sample{y}{i})=p_{ij},i,2,\cdots,n;j=1,2,\cdots,m
\end{eqnarray}
条件熵$H(Y|X)$为已知随机变量$X$的条件下随机变量$Y$的不确定性,随机变量$X$给定的条件下随机变量$Y$的条件熵$H(Y|X)$定义如下
\begin{eqnarray}
H(Y|X)=\sum_{i=1}^np_iH(Y|X=x_i)
\end{eqnarray}
其中,$p_i=P(X=x_i),i=1,2,\cdots,n$。

信息增益(information gain)表示得知特征$X$的信息,是的类$Y$的信息的不确定性减少的程度,定义如下
\paragraph{信息增益}特征$A$对训练数据集$D$的信息增益$g(D,A)$,定义为集合$D$的经验熵$H(D)$与特征$A$给定条件下$D$的经验条件熵$H(D|A)$之差,即
\begin{eqnarray}
g(D,A)=H(D)-H(D|A)
\end{eqnarray}

根据信息增益准则的特征选择方法是:对训练数据集(或子集)$D$,计算其每个特征的信息增益,并比较它们的大小,选择信息增益最大的特征。

设训练数据为$D$,$|D|$表示其样本容量,即样本个数,设有$K$个类$C_k$,$k=1,2,\cdots,K$,$|C_k|$为属于类$C_k$的样本个数,$\sum_{k=1}^K|C_k|=|D|$。设特征$A$有$n$个不同的取值$\{ a_1,a_2,\cdots,a_n \}$,记特征集为$A$,根据某一特征$a$的取值将$D$划分为$n$个子集$D_1,D_2,\cdots,D_n$,$|D_i|$为$D_i$的样本个数,$\sum_{i=1}^n|D_i|=|D|$。记子集$D_i$中属于类$C_k$的样本的集合为$D_{ik}$,即$D_{ik}=D_i\cap C_k$,$|D_{ik}|$为$D_{ik}$的样本个数,信息增益算法如下
\begin{itemize}
\item 输入:训练数据集$D$和特征$a$;
\item 输出:特征$a$对训练数据集$D$的信息增益$g(D,a)$
\item[1] 计算数据集$D$的经验熵$H(D)$
\begin{eqnarray}
H(D)=-\sum_{k=1}^K \frac{|C_k|}{|D|}\log_2\frac{|C_k|}{|D|}
\end{eqnarray}
\item[2] 计算特征$a$对数据集$D$的经验条件熵$H(D|a)$
\begin{eqnarray}
\begin{aligned}
H(D|a)&=\sum_{i=1}^n \frac{|D_i|}{|D|}H(D|a=a_i)\\
&=\sum_{i=1}^n\frac{|D_i|}{|D|}H(D_i)\\
&=-\sum_{i=1}^n\frac{|D_i|}{|D|}\sum_{k=1}^K\frac{|D_{ik}|}{|D_i|}\log_2\frac{|D_{ik}|}{|D_i|}
\end{aligned}
\end{eqnarray}
\item[3] 计算信息增益
\begin{eqnarray}
g(D,A)=H(D)-H(D|a)
\end{eqnarray}
\end{itemize}

于是,在候选属性集合$A$中,选择使得划分后信息增益最大的属性作为最优划分属性,即
\begin{eqnarray}
a_*=\arg\max_{a\in A}g(D,a)
\end{eqnarray}

该算法天生偏向选择分支多的属性,容易导致过拟合。

\subsubsection{信息增益比}
\paragraph{信息增益比}特征$a$对训练数据集$D$的信息增益比$g_R(D,a)$定义为其信息增益$g(D,a)$与训练数据集$D$关于特征$a$的值的熵$H_a(D)$之比,即
\begin{eqnarray}
g_R(D,a)=\frac{g(D,a)}{H_a(D)}
\end{eqnarray}
其中,$H_a(D)=-\sum_{i=1}^n \frac{|D_i|}{|D|}\log_2\frac{|D_i|}{|D|}$,$n$是特征$a$取值的个数。

于是,在候选属性集合$A$中,选择使得划分后信息增益最大比的属性作为最优划分属性,即
\begin{eqnarray}
a_*=\arg\max_{a\in A}g_R(D,a)
\end{eqnarray}

\subsubsection{基尼指数}
分类问题中,假设有$K$个类,样本点属于第$k$类的概率为$p_k$,则概率分布的基尼指数定义为
\begin{eqnarray}
Gini(p)=\sum_{k=1}^K p_k(1-p_k)=1-\sum_{k=1}^K p_k^2
\end{eqnarray}
对于样本集合$D$,$D$的纯度可以用Gini指数来度量
\begin{eqnarray}
Gini(D)=1-\sum_{k=1}^K \left( \frac{|C_k|}{|D|} \right)^2
\end{eqnarray}
其中,$C_k$是$D$中属于第$k$类的样本子集,$K$是类的个数。直观上,$Gini(D)$反映了$D$中随机抽取两个样本,其类别标记不一致的概率。因此,$Gini(D)$越小,则数据集$D$的纯度越高。

设特征$a$有$n$个不同的取值$\{ a_1,a_2,\cdots,a_n \}$,根据特征$a$的取值将$D$划分为$n$个子集$D_1,D_2,\cdots,D_n$,$|D_i|$为$D_i$的样本个数,$\sum_{i=1}^n|D_i|=|D|$,则属性$a$的Gini指数定义为
\begin{eqnarray}
Gini(D,a)=\sum_{i=1}^n \frac{|D_i|}{|D|}Gini(D_i)
\end{eqnarray}
性$a$的Gini指数$Gini(D,a)$表示经$a$分后集合$D$的不确定性,则$Gini$指数值越大,样本集合的不确定性就越大。

于是,在候选属性集合$A$中,选择使得划分后$Gini$指数最小的属性作为最优划分属性,即
\begin{eqnarray}
a_*=\arg\min_{a\in A}Gini(D,a)
\end{eqnarray}

\subsection{决策树的生成}
\subsubsection{ID3}
从根节点开始,对结点计算所有可能的特征的信息增益,选择信息增益最大的特征作为节点的特征,由该特征的不同区直建立子节点,再递归地使用以上方法,构造决策树,直到所有特征的信息增益均最小或没有特征可以选择为止,最后得到一个决策树
\paragraph{ID3算法}
\begin{itemize}
\item 输入:训练数据集$D$,特征集$A$,阈值$\epsilon$
\item 输出:决策树$T$
\item[1] 若$D$中所有实例属于同一类$C_k$,则$T$为单节点树,并将类$C_k$作为该结点的类标记,返回$T$
\item[2] 若$A=\emptyset$,则$T$为单节点树,并将$D$中实例数最大的类$C_k$作为该结点的类标记,返回$T$
\item[3] 否则,计算$A$中各特征对$D$的信息增益$g(D,A_i)$,选择信息增益最大的特征$A_g$
\item[4] 如果$A_g$的信息增益小于阈值$\epsilon$,则置$T$为单结点树,并将$D$中实例数最大的类$C_k$作为该节点的类标记,返回$T$
\item[5] 否则,对$A_g$的每一个可能取值$a_i$,依$a_i$将$D$分割为若干非空子集$D_i$,将$D_i$中实例数最大的类作为标记,构建子结点,由节点及其子结点构成树$T$,返回$T$
\item[6] 对第$i$个子结点,以$D_i$为训练集,以$A-\{A_g\}$为特征集,递归地调用$1\sim 5$,得到子树$T_i$,返回$T_i$
\end{itemize}

\subsubsection{C4.5}
C4.5采用信息增益比来选择特征
\paragraph{C4.5算法}
\begin{itemize}
\item 输入:训练数据集$D$,特征集$A$,阈值$\epsilon$
\item 输出:决策树$T$
\item[1] 若$D$中所有实例属于同一类$C_k$,则$T$为单节点树,并将类$C_k$作为该结点的类标记,返回$T$
\item[2] 若$A=\emptyset$,则$T$为单节点树,并将$D$中实例数最大的类$C_k$作为该结点的类标记,返回$T$
\item[3] 否则,计算$A$中各特征对$D$的信息增益比$g(D,A_i)$,选择信息增益比最大的特征$A_g$
\item[4] 如果$A_g$的信息增益比小于阈值$\epsilon$,则置$T$为单结点树,并将$D$中实例数最大的类$C_k$作为该节点的类标记,返回$T$
\item[5] 否则,对$A_g$的每一个可能取值$a_i$,依$a_i$将$D$分割为若干非空子集$D_i$,将$D_i$中实例数最大的类作为标记,构建子结点,由节点及其子结点构成树$T$,返回$T$
\item[6] 对第$i$个子结点,以$D_i$为训练集,以$A-\{A_g\}$为特征集,递归地调用$1\sim 5$,得到子树$T_i$,返回$T_i$
\end{itemize}

\subsubsection{CART}
对回归树用平方误差最小化准则,对分类树用Gini指数最小化准则,进行特征选择,生成二叉树。

1. 回归树的生成

假设$X$和$Y$分别为输入和输出变量,并且$Y$是连续变量,给定训练数据集$ D=\{ (\sample{x}{1},\sample{y}{1}),(\sample{x}{2},\sample{y}{2}),\cdots,(\sample{x}{N},\sample{y}{N}) \} $。回归树将输入空间划分为$M$个单元$D_1,D_2,\cdots,D_M$,且在每个单元上有一个固定值$c_m$,因此回归树模型表示为
\begin{eqnarray}
f(x)=\sum_{m=1}^M c_m1(x\in D_m)
\end{eqnarray}
其中,$1(x)$为示性函数。具体的,为求解
\begin{eqnarray}
\min_{j,s}
\left(
	\min_{c_1}\sum_{x_i\in D_1(j,s)}(y_i-c_1)^2+\min_{c_2}\sum_{x_i\in D_2(j,s)}(y_i-c_2)^2
\right)
\end{eqnarray}
其中,$j$为最优划分变量,$s$为最优划分点。$D_1(j,s)=\{x|x_j\leq s\}$,$D_2(j,s)=\{x|x_j> s\}$。对于$j$和$s$的选取,采用遍历的方法。遍历划分变量$j$,再以步长$\Delta s$扫描划分点$s$。对于$j$,$s$都固定,且用平方误差$\sum_{x_i\in D_m}(\sample{y}{i}-f(\sample{x}{i}))^2$来表示回归树对于训练数据的预测误差,则可求得在单元$D_m$上的$c_m$的最优值$\hat{c_m}$为
\begin{eqnarray}
\hat{c_m}=\frac{1}{N_m}\sum_{x_i\in D_m(j,s)}y_i,\ m=1,2
\end{eqnarray}
其中,$N_m$为单元$D_m$中的样本个数。其表示对$D_m$的所有样本的$y$取均值。

经过两轮遍历之后,即可选出最优划分变量和最优划分点,以及计算出对应的$\hat{c_m}$。算法如下

\paragraph{最小二乘回归树生成算法}
\begin{itemize}
\item 输入:训练数据集$D$,特征集$A$,阈值$\epsilon$
\item 输出:回归树$f(x)$
\item[1] 若$D$中所有实例的输出均为$y_0$,则$T$为单节点树,并将$y_0$作为该结点的输出值,返回$T$
\item[2] 否则,采用遍历的方法。遍历划分变量$j$,对于固定的划分变量$j$再以步长$\Delta s$扫描划分点$s$,对于$j$,$s$都固定,求$c_1$,$c_2$
\begin{eqnarray}
c_m=\frac{1}{N_m}\sum_{x_i\in D_m(j,s)}y_i,\ m=1,2
\end{eqnarray}
来求解
\begin{eqnarray}
\min_{j,s}
\left(
	\min_{c_1}\sum_{x_i\in D_1(j,s)}(y_i-c_1)^2+\min_{c_2}\sum_{x_i\in D_2(j,s)}(y_i-c_2)^2
\right)
\end{eqnarray}
并记损失函数
\begin{eqnarray}
L=\sum_{x_i\in D_1(j,s)}(y_i-c_1)^2+\sum_{x_i\in D_2(j,s)}(y_i-c_2)^2
\end{eqnarray}
\item[3] 如果$L$小于阈值$\epsilon$,则置$T$为单结点树,并将$D$中实例的输出的均值$\hat{c}=\frac{1}{N_{|D|}}\sum_{x_i\in D}y_i$作为该结点的输出值,返回$T$
\item[4] 否则,用选定的对$(j,s)$划分区域,并决定相应的输出值
\begin{eqnarray}
D_1(j,s)&=&\{x|x_j\leq s\}\\
D_2(j,s)&=&\{x|x_j> s\}\\
\hat{c_m}&=&\frac{1}{N_m}\sum_{x_i\in D_m(j,s)}y_i,\ m=1,2
\end{eqnarray}
\item[5] 对这两个子区域$D_i,\ i=1,2$,以$D_i$为训练集,以$A$为特征集,递归地调用$1\sim 5$,得到子树$T_i$,返回$T_i$
\item[6] 将输入空间划分为$M$个区域$D_1,D_2,\cdots,D_M$,生成决策树
\begin{eqnarray}
f(x)=\sum_{m=1}^M c_m1(x\in D_m)
\end{eqnarray}
\end{itemize}


注:对比ID3和C4.5算法,由于回归树少了将特征集进行剔除,即少了第5步以$A-\{A_g\}$为特征集,因而少了这一步:
\begin{itemize}
\item[2] 若$A=\emptyset$,则$T$为单节点树,并将$D$中实例的输出的均值$\hat{c}=\frac{1}{N_{|D|}}\sum_{x_i\in D}y_i$作为该结点的输出值,返回$T$
\end{itemize}

