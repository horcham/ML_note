\section{指数函数族与广义线性模型}
参数为$\eta$的变量$x$的指数族分布定义为
\begin{eqnarray}
p(x|\eta)=h(x)g(\eta)e^{\eta^Tu(x)}
\end{eqnarray}
其中,$x$可为标量或向量,离散或连续的。
\subsection{伯努利分布导出logistic回归}
考虑伯努利分布,其为二元变量的取值分步。可通过该分布导出logistic函数。伯努利分布及其指数式变形如下
\begin{eqnarray}
\begin{aligned}
p(x|\eta)&=\textbf{Bern}(x|\eta)\\
&=\mu^x(1-\mu)^{1-x}\\
&=e^{x\ln\mu+(1-x)\ln(1-\mu)}\\
&=(1-\mu)e^{\ln(\frac{\mu}{1-\mu})x}
\end{aligned}
\end{eqnarray}
令$h(x)=1$,$g(\eta)=1-\eta$,$\eta^T=\ln(\frac{\mu}{1-\mu})$,$u(x)=x$,因而有
\begin{eqnarray}
\mu=\ln(\frac{\mu}{1-\mu})
\end{eqnarray}
其反函数为
\begin{eqnarray}
\sigma(\eta)=\frac{1}{1+e^{-\eta}}
\end{eqnarray}
因而,logistic回归的形式为
\begin{eqnarray}
y=\sigma(\theta^Tx)
\end{eqnarray}
\subsection{多项式分布导出softmax回归}
考虑多项式分布,其有$M$个可能取值,$\mu_i$为当$x=i$时的概率,因而有
\begin{eqnarray}
\sum_{k=1}^M=1
\end{eqnarray}
定义指示函数
\begin{eqnarray}
I=
\left\lbrace
\begin{aligned}
1,\textbf{True}\\
0,\textbf{False}
\end{aligned}
\right.
\end{eqnarray}
则多项式分布可表示为
\begin{eqnarray}
\begin{aligned}
p(x|\mu) &= \prod_{k=1}^M\mu_k^{I(x_k=1)}\\
&= \exp
	\left\lbrace
	\begin{aligned}
	{\sum_{k=1}^Mx_k\ln\mu_k}\\
	\end{aligned}
	\right\rbrace\\
&= \exp
	\left\lbrace
	\begin{aligned}
	{\sum_{k=1}^{M-1}x_k\ln\mu_k+(1-\sum_{k=1}^{M-1}x_k)\ln(1-\sum_{k=1}^{M-1}\mu_k)}
	\end{aligned}
	\right\rbrace\\
&= \exp
	\left\lbrace
	\begin{aligned}
	\sum_{k=1}^{M-1}x_k\ln(\frac{\mu_k}{1-\sum_{j=1}^{M-1}})+\ln(1-\sum_{k=1}^{M-1}\mu_k)
	\end{aligned}
	\right\rbrace
\end{aligned}
\end{eqnarray}
其为指数族函数,其中
\begin{eqnarray}
\begin{aligned}
h(x)&=1\\
g(\eta)&=1-\sum_{k=1}^{M-1}\mu_k\\
\eta &= 
\begin{pmatrix}
\frac{\mu_1}{1-\sum_{j=1}^{M-1}}\\
\frac{\mu_2}{1-\sum_{j=1}^{M-1}}\\
\vdots\\
\frac{\mu_{M-1}}{1-\sum_{j=1}^{M-1}}
\end{pmatrix}\\
u(x)&=x
\end{aligned}
\end{eqnarray}
令
\begin{eqnarray}
\ln
\left(
\begin{aligned}
\frac{\mu_k}{1-\sum_j\mu_j}
\end{aligned}
\right)
=\eta_k
\end{eqnarray}
求其反函数,有
\begin{eqnarray}
\mu_k=\frac{\exp(\eta_k)}{1+\sum_j^{M-1}\exp(\eta_j)}
\end{eqnarray}
该函数为logistic函数在多元情况下的拓展,称为softmax函数或归一化指数。采用广义线性模型的想法,有
\begin{eqnarray}
p(C_K|x)=y_k=\frac{e^{\theta_k^Tx}}{\sum_je^{\theta_k^Tx}}
\end{eqnarray}
对于第$n$个样本,有
\begin{eqnarray}
\prod_{k=1}^Kp(C_K|x_n)^{I(x_k=1)}=y_k^{I(x_k=1)}
\end{eqnarray}
所以其似然函数为
\begin{eqnarray}
L(\theta)= \prod_{n=1}^N\prod_{k=1}^Kp(C_K|x_n)^{I(x_{nk}=1)}=\prod_{n=1}^N\prod_{k=1}^Ky_{nk}^{I(x_{nk}=1)}
\end{eqnarray}