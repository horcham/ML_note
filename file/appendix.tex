\section{附录}
\subsection{SVD分解}
假设$A$是一个$m\times n$的矩阵,可以分解为如下形式
\begin{eqnarray}
A = U\Sigma V^T
\end{eqnarray}
其中,$V$为$n\times n$的矩阵,如下定义
\begin{eqnarray}
(A^TA)v_i=\lambda_iv_i
\end{eqnarray}
$\lambda_i$为$A^TA$的特征值,$v_i$为$A^TA$的特征向量(为列向量)。$\Sigma$为$m\times n$的对角矩阵,对角元素为如下定义
\begin{eqnarray}
\Sigma_{ii}=\sqrt{\lambda_i}
\end{eqnarray}
$U$为$m\times m$的矩阵,定义为
\begin{eqnarray}
u_i=\frac{1}{\Sigma_{ii}}Av_i
\end{eqnarray}

上述$\sigma_i=\Sigma_{ii}$为奇异值,$u$为左奇异向量,奇异值在$\Sigma$中排列为从大到小,因此$\sigma$会减小地很快。多数情况下$10\%$甚至$1\%$的奇异值的和已经占了全部奇异值的和的$99\%$以上,因而可以选择前$r$个奇异值来近似地描述矩阵,定义\textbf{部分奇异值分解}:
\begin{eqnarray}
A_{m\times n}\approx U_{m\times r}\Sigma_{r\times r}V^T_{r\times n}
\end{eqnarray}
其中,$r$是一个远小于$m,n$的数。对于很大的矩阵$A$,储存将会耗费很大的空间,反而储存$U,S,V$是一个很好的选择。

\subsection{范数}
\subsubsection{常用向量范数}
对于$x=\vectornew{x}{n}^T$,则
\paragraph{p-范数}$||x||_p = (\sum_i^n |x_i|^p)^{\frac{1}{p}}$
\paragraph{1-范数}$||x||_1 = \sum_i^n |x_i|$
\paragraph{2-范数}$||x||_2 = (\sum_i^n |x_i|^2)^{\frac{1}{2}}$
\paragraph{$\infty$-范数}$||x||_p = \max\{|x_1|,\cdots,|x_n|\}$
\subsubsection{常用矩阵范数}
\paragraph{p-范数}$||A||_1=\max\{\sum|a_{i1}|,\sum|a_{i2}|,\cdots,\sum|a_{in}|\}$(列和范数,$A$每一列元素绝对值之和的最大值),其中$\sum|a_{i1}|$每一列元素绝对值的和$\sum|a_{i1}|=|a_{11}|+|a_{21}|+\cdots+|a_{n1}|$。
\paragraph{2-范数}$||A||_2=A$的最大奇异值$=(\max\{ \lambda_i(A^H*A) \})^\frac{1}{2}$(谱范数,即$A^H*A$特征值$\lambda_i$的平方根,其中$A^H$为$A$的共轭转置矩阵)
\paragraph{$\infty$-范数}$||A||_\infty=\max\{\sum|a_{1j}|,\sum|a_{2j}|,\cdots,\sum|a_{mj}|\}$(行和范数,$A$每一行元素绝对值之和的最大值),其中$\sum|a_{1j}|$每一列元素绝对值的和$\sum|a_{1j}|=|a_{11}|+|a_{12}|+\cdots+|a_{1m}|$